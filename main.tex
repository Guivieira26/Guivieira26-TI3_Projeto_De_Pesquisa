\documentclass{beamer}

\usepackage[utf8]{inputenc}
\usepackage[brazil]{babel}
\usepackage{graphicx}
\title{Análise de Sentimentos em Comentários de Produtos Online}
\author{Guilherme Henrique Vieira Nascimento}
\institute{Pontifícia Universidade Católica}
\date{\today}

\begin{document}

% Slide de Título
\begin{frame}
    \begin{figure}
		\includegraphics[width=0.25\linewidth]{img/logo_puc.png}
	\end{figure}
    \titlepage
\end{frame}

% Slide de Resumo
\begin{frame}{Resumo}
    \begin{itemize}
        \item \textbf{Contexto:} Desenvolver um sistema de análise de sentimentos para comentários de produtos online.
        \item \textbf{Justificativa:} Necessidade de identificar rapidamente a opinião dos usuários sobre os produtos.
        \item \textbf{Objetivo:} Criar um sistema de análise automatizada de comentários para auxiliar empresas e consumidores.
        \item \textbf{Metodologia:} Coleta de dados, pré-processamento e aplicação de algoritmos de NLP.
    \end{itemize}
\end{frame}

% Slide de Introdução
\begin{frame}{Introdução}
    \begin{itemize}
        \item \textbf{Contextualização:} Explicar a relevância de análises automáticas de comentários em plataformas de e-commerce.
        \item \textbf{Problema:} Dificuldade das empresas em monitorar manualmente grandes volumes de avaliações.
        \item \textbf{Justificativa:} Um sistema automatizado facilita a tomada de decisões estratégicas.
        \item \textbf{Metodologia:} Utilizar técnicas de NLP para identificar padrões de sentimentos nos textos.
    \end{itemize}
    \textbf{Organização do Trabalho:}
    \begin{itemize}
        \item \textbf{Seção 2:} Referencial teórico usado no trabalho.
        \item \textbf{Seção 3:} Procedimentos metodológicos.
    \end{itemize}
\end{frame}

% Slide de Objetivos
\begin{frame}{Objetivos}
    \textbf{Objetivo Geral:} Desenvolver um sistema de análise de sentimentos para comentários de produtos.
    
    \vspace{0.5cm}
    \textbf{Objetivos Específicos:}
    \begin{itemize}
        \item Coletar e pré-processar dados de comentários.
        \item Implementar um modelo de aprendizado de máquina para classificação de sentimentos.
    \end{itemize}
\end{frame}

% Slide de Revisão Bibliográfica
\begin{frame}{Revisão Bibliográfica}
    \textbf{Fundamentação Teórica:}
    \begin{itemize}
        \item Conceitos de processamento de linguagem natural e aprendizado de máquina.
    \end{itemize}
    
    \vspace{0.5cm}
    \textbf{Trabalhos Relacionados:}
    \begin{itemize}
        \item Comparar com estudos similares, discutindo resultados obtidos e suas limitações.
    \end{itemize}
\end{frame}

% Slide de Metodologia
\begin{frame}{Metodologia}
    \textbf{Tipo de Pesquisa:} Exploratória, qualitativa e quantitativa.
    
    \vspace{0.5cm}
    \textbf{Etapas:}
    \begin{itemize}
        \item Coleta de dados.
        \item Pré-processamento dos dados.
        \item Treinamento do modelo de NLP.
        \item Avaliação e validação dos resultados.
    \end{itemize}
\end{frame}

% Slide de Cronograma
\begin{frame}{Cronograma}
    \textbf{Apresentação do Cronograma:}
    
    \begin{table}[]
        \begin{tabular}{|c|c|c|c|c|c|c|c|c|c|c|c|c|}
            \hline
            Atividades & Jan & Fev & Mar & Abr & Mai & Jun & Jul & Ago & Set & Out & Nov & Dez \\ \hline
            Revisão Bibliográfica & X & X & & & & & & & & & & \\ \hline
            Coleta de Dados & & X & X & & & & & & & & & \\ \hline
            Desenvolvimento & & & X & X & X & & & & & & & \\ \hline
            Testes e Validação & & & & & X & X & X & & & & & \\ \hline
            Redação Final & & & & & & & X & X & X & & & \\ \hline
        \end{tabular}
    \end{table}
\end{frame}

% Slide de Referências
\begin{frame}{Referências}
    \begin{thebibliography}{99}
        \bibitem{artigo1} Carmo Neto, Francisco Vieira. \textit{Análise de sentimentos de reviews de produtos de e-commerces brasileiros}. UFAL, 2022.
        \bibitem{artigo2} Carlos Levi Da Silva Albuquerque. \textit{ANÁLISE DE SENTIMENTO SOBRE COMENTÁRIOS EM SITES DE E-COMMERCE
NO IDIOMA PORTUGUÊS/BR}. Unichristus, 2022.
    \end{thebibliography}
\end{frame}

\end{document}
